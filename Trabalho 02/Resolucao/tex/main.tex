\documentclass[12pt]{article}

\usepackage{sbc-template}
\usepackage[T1]{fontenc}
\usepackage[portuguese]{babel}

\usepackage{mathtools}
\usepackage{amssymb}
\usepackage{amsmath}

\usepackage{url}

\usepackage{import}
\usepackage{listings}

\usepackage{graphicx,url}
\usepackage{xcolor}
\usepackage[utf8]{inputenc}  

\usepackage{algorithm}
\usepackage[noend]{algpseudocode}

\usepackage[edges]{forest}

\definecolor{codegreen}{rgb}{0,0.6,0}
\definecolor{codegray}{rgb}{0.5,0.5,0.5}
\definecolor{codepurple}{rgb}{0.58,0,0.82}
\definecolor{backcolour}{rgb}{0.95,0.95,0.92}

\lstdefinestyle{codigo}{
    commentstyle=\color{codegreen},
    keywordstyle=\color{magenta},
    numberstyle=\tiny\color{codegray},
    stringstyle=\color{codepurple},
    basicstyle=\ttfamily\footnotesize,
    numberstyle=\tiny,
    basicstyle=\ttfamily\footnotesize,
    breakatwhitespace=false,         
    breaklines=true,                 
    captionpos=b,                    
    keepspaces=true,                 
    numbers=left,                    
    numbersep=5pt,                  
    showspaces=false,                
    showstringspaces=false,
    showtabs=false,                  
    tabsize=2
}

\lstset{
    style=codigo,
    inputencoding=utf8,
    extendedchars=false,  % Extended ASCII
}

\graphicspath{{img/}}
     
\sloppy

\title{Uma argumentação entre \emph{Backtracking} e \emph{Branch and Bound}}

\author{Gustavo Lopes Rodrigues\inst{1}, Rafael Amauri Diniz Augusto\inst{2}}

\address{Instituto de Ciências Exatas e Informática -- \\
Pontifícia Universidade Católica de Minas Gerais(PUC-MG)}

\begin{document} 

  \maketitle

  \import{./sections}{abstract.tex}

  \import{./sections}{general-information.tex}

  \import{./sections/main-estrategies}{backtracking.tex}

  \import{./sections/main-estrategies}{branch-and-bound.tex}

  \import{./sections/main-estrategies}{comparison.tex}

  \section{Outras estratégias} \label{sec:other-strategies}

    Com as principais estratégias discutidas, fica claro as diferenças 
    e vantagens de ambas estratégias. Agora iremos discutir outras possíveis
    abordagens para resolução de algoritmos, e comparar com as já discutidas.
    
    \import{./sections/other-estrategies}{greedy-algorithm.tex}

    \import{./sections/other-estrategies}{divide-and-conquer.tex}
  
    \import{./sections/other-estrategies}{dynamic-programming.tex}

  \section{Conclusão}

  No fim das contas, \emph{Backtracking} e \emph{Branch and Bound} são estratégias 
  bem semelhantes, porém, como foi demonstrado neste artigo, eles não substituem um ao 
  outro. Devido a isso, cabe ao programador analisar na qual este quer fazer a implementação 
  de um algoritmo de otimização, pensando qual são as necessidades do contexto atual.

  Isso também vale para as outras estratégias mencionadas, todas elas possuem suas aplicações 
  específicas, por mais que possuem propriedades semelhantes umas as outras.

  \bibliographystyle{sbc}
  \bibliography{sbc-template}

\end{document}
