\section{Backtracking} \label{sec:backtracking}

\emph{Backtracking} é uma estratégia para resolver problemas recursivamente e incrementalmente,
removendo as soluções parciais que falham em ajudar na solução para o problema. Para aplicar essa
técnica, o algoritmo tenta encontrar a solução utilizando diversos pequenos \emph{checkpoints}, para os
quais o programa pode voltar se a iteração atual para a resolução do problema não ajudar a encontrar a
solução final.

Essa estratégia é viável para resolver problemas modulares que requerem muita tentativa e erro, já que ele
remove “caminhos” inválidos, e isso salva muito tempo de processamento.

\subsection{Problema do labirinto}

Considere o labirinto abaixo:

\begin{figure}[ht]
  \centering
  \includegraphics[width=.3\textwidth]{labirinto.jpg}
  \caption{Um exemplo de labirinto}
  \label{fig:labirinto}
\end{figure}

Resolver labirintos é uma aplicação clássica da estratégia de \emph{backtracking}, pois envolve diversas
possibilidades de caminhos e é fácil integrar \emph{checkpoints} no problema.
Imaginando que podemos traduzir a imagem acima para uma matriz onde cada ponto de decisão do
labirinto é uma célula, que todas células têm ponteiros que apontam para quatro direções: cima, baixo,
esquerda e direita, e que esses ponteiros podem levar a outras células ou ser do tipo NULL (indicando que
não há uma célula ligada naquela direção), a estratégia de backtracking pode ser usada para resolver o
problema.

\newpage