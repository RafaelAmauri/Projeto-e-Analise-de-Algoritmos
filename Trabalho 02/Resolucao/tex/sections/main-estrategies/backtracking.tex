\section{Backtracking} \label{sec:backtracking}

\emph{Backtracking} é uma estratégia para resolver problemas recursivamente e incrementalmente,
removendo as soluções parciais que falham em ajudar na solução para o problema. Para aplicar
essa técnica, a estratégia tenta encontrar a solução utilizando diversos pequenos \emph{checkpoints},
para os quais o programa pode voltar se a iteração atual para a resolução do problema não ajudar a
encontrar a solução final.

Essa estratégia é viável para resolver problemas modulares que requerem muita tentativa e erro, 
já que ele remove “caminhos” inválidos, e isso salva muito tempo de processamento.

\subsection{Problema do labirinto}

Considere o labirinto abaixo:

\begin{figure}[ht]
  \centering
  \includegraphics[width=.3\textwidth]{labirinto.jpg}
  \caption{Um exemplo de labirinto}
  \label{fig:labirinto}
\end{figure}

Resolver labirintos é uma aplicação clássica da estratégia de \emph{backtracking},
pois envolve diversas possibilidades de caminhos e é fácil integrar \emph{checkpoints} no problema.

Imaginando que podemos traduzir a imagem acima para uma matriz onde cada ponto de 
decisão do labirinto é uma célula, que todas células têm ponteiros que apontam para 
quatro direções: cima, baixo, esquerda e direita, e que esses ponteiros podem levar 
a outras células ou ser do tipo NULL (indicando que não há uma célula ligada naquela direção),
a estratégia de Backtracking pode ser usada para resolver o problema da seguinte forma:

\begin{figure}[ht]
  \centering
  \includegraphics[width=.3\textwidth]{labirinto-resolvido.jpg}
  \caption{O mesmo labirinto de antes, porém com sua solução destacada}
  \label{fig:labirinto-resolvido}
\end{figure}

Admitindo que esse algoritmo busca os caminhos disponíveis na ordem [cima, esquerda, baixo, direita],
os caminhos vermelhos são caminhos percorridos pelo algoritmo e que foram detectados como caminhos
que não contribuem para a solução. A ideia é preservar os checkpoints “certos” que foram percorridos
e voltar ao último checkpoint correto ao invés de salvar os caminhos “errados” e recomeçar o percurso
desde o início. Dessa forma, apenas os checkpoints “certos” vão ser mantidos e o caminho correto
eventualmente será encontrado, sem a necessidade de recalcular e levar em conta os caminhos que 
já sabemos que são incorretos.

\begin{algorithm}
  \caption{Backtracking}
  \begin{algorithmic}[1]
  \Procedure{find\_path}{x,y}
  \State {$\text{grid[x][y] == visited}$}
  \For{$\text{i \textbf{to} 4}$}
  \If{$\text{UP(x,y+1)}$}
  \State {$\textbf{find\_path(x,y+1)}$}
  \EndIf
  \If{$\text{LEFT(x-1,y)}$}
  \State {$\textbf{find\_path(x-1,y)}$}
  \EndIf
  \If{$\text{DOWN(x,y-1)}$}
  \State {$\textbf{find\_path(x,y-1)}$}
  \EndIf
  \If{$\text{RIGHT(x+1,y)}$}
  \State {$\textbf{find\_path(x+1,y)}$}
  \EndIf
  \EndFor
  \EndProcedure
  \end{algorithmic}
\end{algorithm}

\nocite{backtracking}

\newpage