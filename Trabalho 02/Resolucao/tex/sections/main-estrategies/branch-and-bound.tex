\section{Branch and Bound} \label{sec:branch-and-bound}

O método de Ramificar e limitar (em inglês, Branch and bound) é um algoritmo para 
encontrar soluções ótimas para vários problemas de otimização, especialmente em 
otimização combinatória. Consiste em uma enumeração sistemática de todos os candidatos
a solução, através da qual grandes subconjuntos de candidatos infrutíferos são 
descartados em massa utilizando os limites superior e inferior da quantia otimizada.

O método foi proposto por A. H. Land e A. G. Doig em 1960 para programação discreta.
É utilizado para vários problemas NP-completos como o problema do caixeiro viajante
e o problema da mochila. 

\subsection{Atribuição de trabalho}


\begin{table}[ht]
    \centering 
    \begin{tabular}{|c | c | c | c |} 
        \hline
         & \textbf{Job 1} & \textbf{Job 2} & \textbf{Job 3} \\ 
        \hline
        \textbf{A} & 9 & 3 & 4 \\ 
        \hline
        \textbf{B} & 7 & 8 & 4 \\
        \hline
        \textbf{C} & 10 & 5 & 2 \\
        \hline
   \end{tabular}
\end{table}