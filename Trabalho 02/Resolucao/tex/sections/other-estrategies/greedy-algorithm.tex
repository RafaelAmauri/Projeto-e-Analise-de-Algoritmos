\subsection{Abordagem gulosa}

    Algoritmo guloso é a estratégia que tenta resolver o problema fazendo a 
    escolha localmente ótima em cada fase com a esperança de encontrar 
    um ótimo global que resolve todo o problema. 

    Na solução de alguns problemas combinatórios a estratégia gulosa pode
    assegurar a obtenção de soluções ótimas, o que não é muito comum. No
    entanto, quando o problema a ser resolvido pertencer à classe NP-completo
    ou NP-difícil, a estratégia gulosa torna-se atrativa para a obtenção de
    solução aproximada em tempo polinomial.

    Um exemplo de abordagem gulosa é o Algoritmo de Dijkstra.

\subsubsection{Algoritmo de Dikstra}

    Dado um grafo qualquer, e dois pontos: partida e chegada, encontre o menor caminho possível.
    Para resolver esse problema, Dijkstra cria o que é chamado de \emph{Shortest path tree}(spt) ou 
    a árvore de caminho mais curto. A cada passo do algoritmo a \textbf{spt} adiciona uma vértice não visitada,
    e então atualiza a distância das adjacências da vértice que foi adicionada. O algoritmo irá terminar, quando 
    a \textbf{spt} conter informação de todas as vértices.