
  \subsection{Programação Dinâmica}

    Programação dinâmica é uma estratégia de construção de algoritmos que 
    possui uma grande eficácia na resolução de problemas de otimização combinatória.
    Em outras palavras, essa estratégia é ideal quando o problema original 
    é divido em subproblemas, que se repetem, logo podem ser memorizados para 
    evitar recálculos.

    Antes de aplicar essa técnica, o problema precisa atender duas propriedades:
    um problema com subproblemas \textbf{(Overlapping subproblems)}, propriedade já vista 
    em Divisão e conquista e uma subestrutura ideal \textbf{(Optimal substructure)}, propriedade 
    já vista em algoritmos gulosos.

    Para melhor exemplificar esses conceitos, vamos analisar um clássico problema da computação, 
    e também um caso perfeito para ilustrar a estratégia de programação dinâmica.

    \subsubsection{Fibonacci}

    Descoberto pelo matemático italiano do século 12, Leonardo Fibonacci, a sequência 
    Fibonacci é uma sucessão de números que aparece codificada nos mais diversos fenômenos 
    da natureza(espiral de galáxias, ciclones, conchas e vários outros). Na computação, temos 
    o problema de criar um algoritmo capaz de calcular o número da sequência de Fibonacci.

    \begin{figure}[ht]
      \centering
      \includegraphics[width=.3\textwidth]{fibonnaci.jpg}
      \caption{Uma imagem ilustrativa da sequência Fibonacci}
      \label{fig:fibonnaci}
    \end{figure}

    A ideia para resolução desse problema é bem simples: sabendo que a sequência inicia com 0 e 1, os
    próximos números serão sempre a soma dos dois números anteriores. A seguir temos um exemplo 
    de uma implementação ingênua do algoritmo da sequência de Fibonacci.

    \begin{algorithm}
      \caption{Naive Fibonacci} 
      \begin{algorithmic}[1]
      \Procedure{fib}{n}
      \If{$\text{n $\geq$ 2}$}
      \State {$\text{\textbf{return} n}$}
      \EndIf
      \State {$\text{\textbf{return} fib(n-1) + fib(n-2)}$}
      \EndProcedure
      \end{algorithmic}
    \end{algorithm}

    Esta implementação a primeira vista não parece ter nenhum problema. Porém, é quando tentamos 
    executar esse código que o problema se mostra. O código consegue calcular o 6°, 7° e 8° número da 
    sequência, porém o código não consegue calcular o 50° número. 
 
    Para entender a natureza desse problema precisamos entender a complexidade desse código, logo,
    precisamos da árvore de recursão gerada pelo código:

    \begin{figure}[ht]
      \centering
      \begin{forest}
        for tree={
            grow=south,
            circle, draw, minimum size=3ex, inner sep=2pt,
            s sep=7mm
                }
        [6
            [5
                [4
                  [3
                    [2][1]
                  ]
                  [2]
                ]
                [3
                    [2][1]
                ]
            ]
            [4
                [3
                    [2][1] 
                ]
                [2]
            ]
        ]
        \end{forest}  
        \caption{encontrando o 6° número da sequência Fibonacci}
    \end{figure}

    A primeira coisa a se notar é a complexidade do algoritmo. A 
    árvore formada possui uma altura N(o número que queremos achar da sequência),
    e a cada novo nível da árvore, formamos no máximo, dois novos nós, logo 
    podemos concluir que a complexidade é de $O(2^{n})$, justificando também, porque 
    não conseguimos calcular o 50° número, já que este levaria um tempo muito alto.

    Então agora a pergunta fica: como podemos otimizar essa árvore, de forma 
    que o algoritmo consiga ter um tempo linear? Voltando à árvore, podemos observar 
    um padrão recorrente.


    \begin{figure}[ht]
      \centering
      \begin{forest}
        for tree={
            grow=south,
            circle, draw, minimum size=3ex, inner sep=2pt,
            s sep=7mm
                }
        [6
            [5
                [4, tikz={\node[draw,circle,black,fit=()(!1),inner sep=0mm,xshift=3mm]{};} 
                  [3, tikz={\node[draw,circle,black,fit=()(!1),inner sep=0mm,xshift=3mm]{};}
                    [2][1]
                  ]
                  [2]
                ]
                [3 , tikz={\node[draw,circle,black,fit=()(!1),inner sep=0mm,xshift=3mm]{};} 
                    [2][1]
                ]
            ]
            [4 , tikz={\node[draw,circle,black,fit=()(!1),inner sep=0mm,xshift=3mm]{};} 
                [3 , tikz={\node[draw,circle,black,fit=()(!1),inner sep=0mm,xshift=3mm]{};} 
                    [2][1] 
                ]
                [2]
            ]
        ]
        \end{forest}  
        \caption{Árvore de recursão, com os sub-problemas repetidos circulados}
    \end{figure}

    Com esse segundo desenho, fica mais evidente que existe passos da recursão que estão repetidos, ou seja,
    o programa acaba resolvendo mais de uma vez os mesmos problemas de forma desnecessária. Em conclusão, para 
    resolver esses problemas, o ideal seria uma estrutura que possa guardar 
    resultados de recursões já feitas para usar depois quando o mesmo problema for encontrado, resultando 
    em uma árvore de recursão da seguinte forma:

    \newpage

    \begin{figure}[ht]
      \centering
      \begin{forest}
        for tree={
            grow=south,
            circle, draw, minimum size=3ex, inner sep=2pt,
            s sep=7mm
                }
        [6
            [5
                [4
                  [3]
                ]
                [3]
            ]
            [4]
        ]
        \end{forest}  
        \caption{Árvore de recursão otimizada}
    \end{figure}

    Com a otimização percebemos que agora a complexidade do 
    código melhorou drasticamente, tendo um resultado de 
    $O(2n)$, visto que é resultado da altura da árvore multiplicado 
    por 2, que pode ser simplificado para apenas $O(n)$.

    Eis então agora o código otimizado:

    \begin{algorithm}
      \caption{Optimized Fibonacci} 
      \begin{algorithmic}[1]
      \Procedure{fib}{n,memo}
      \If{$\text{n $\geq$ 2}$}
      \State {$\text{\textbf{return} n}$}
      \EndIf
      \If{$\text{n \textbf{in} memo}$}
      \State {$\text{\textbf{return} memo[n]}$}
      \EndIf
      \State {$\text{memo $\leftarrow$ fib(n-1) + fib(n-2)}$}
      \State {$\text{\textbf{return} memo[n]}$}
      \EndProcedure
      \end{algorithmic}
    \end{algorithm}

    \subsubsection{Programação dinâmica Vs Backtracking}

    \begin{itemize}
      \item Programação dinâmica procura a solução ótima para o problema, enquanto 
      que backtracking tenta sistemáticamente satisfazer um critério, podendo ou não 
      varrer por toda a árvore, assim como pode ou não encontrar a solução ótima.
      \item Ambos algoritmos possuem conceitos que tem natureza recursiva. Entretanto
      programação dinâmica ainda pode ser implementado de forma iterativa, usando 
      bottom-up (tabulation).
    \end{itemize}

    \subsubsection{Programação dinâmica Vs Branch and Bound}

    \begin{itemize}
        \item Programação dinâmica tenta minimizar os passos para resolver 
        o problema, evitando subproblemas já resolvidos. Enquanto isso, 
        branch-and-bound tenta sistemáticamente enumerar todos os candidatos 
        possíveis na árvore de soluções possíveis,
        \item Branch and Bound permite backtracking, para selecionar as melhores 
        soluções dentro da árvore de estados, enquanto programação dinâmica não utiliza
        esse tipo de recursão. Como resultado, Branch and Bound pode se demonstrar 
        mais lento que programação dinâmica em alguns casos.
    \end{itemize}

    \nocite{dynamic-programming}

\newpage