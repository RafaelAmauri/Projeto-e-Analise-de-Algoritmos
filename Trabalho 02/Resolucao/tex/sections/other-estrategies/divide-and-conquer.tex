\subsection{Divisão e conquista}

    Divisão e Conquista é uma estratégia para projeto de algoritmos utilizada 
    pela primeira vez por Anatolii Karatsuba em 1960. Esta técnica consiste em dividir um problema maior 
    recursivamente em problemas menores, até que o problema possa ser resolvido diretamente. Então a 
    solução do problema inicial é dada através da combinação dos resultados de todos os problemas 
    menores computados. Um exemplo famoso de problemas que usam desta estratégia, é o problema de 
    ordenação interna(quicksort e mergesort), e o problema dos pares de pontos próximos.

\subsubsection{Pares de pontos próximos}

    Dado um conjunto de n pontos em um espaço, encontrar os dois pontos do conjunto que possuem a 
    menor distância um do outro. Este problema possui mais de uma possível implementação, porém 
    uma das mais eficientes é utilizando divisão e conquista.

    Supondo que a entrada para o problema é um conjunto de pontos com o eixo x já ordenado,
    encontramos o ponto do meio, separamos o conjunto ao meio, e então recursivamente descobrimos
    as menores distâncias entre os conjuntos separados.