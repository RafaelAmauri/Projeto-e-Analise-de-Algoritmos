\subsection{Divisão e conquista}

    Divisão e Conquista (do inglês Divide and Conquer) em computação é uma técnica 
    de projeto de algoritmos utilizada pela primeira vez por Anatolii Karatsuba em 1960
    no algoritmo de Karatsuba. 

    Esta técnica consiste em dividir um problema maior recursivamente em
    problemas menores até que o problema possa ser resolvido diretamente. Então a 
    solução do problema inicial é dada através da combinação dos resultados de todos 
    os problemas menores computados. Vários problemas podem ser solucionados através 
    desta técnica, como a ordenação de números através do algoritmo merge sort e a 
    transformação discreta de Fourier através da transformada rápida de Fourier. 
    Outro problema clássico que pode ser resolvido através desta técnica é a Torre de Hanoi. 

\subsubsection{Pares de pontos próximos}